\documentclass[a4paper]{report}
\usepackage[a4paper, margin=20mm]{geometry}
\usepackage{amsmath, pdflscape}
\usepackage[ansinew]{inputenc}
%\usepackage[spanish]{babel}
\usepackage{tikz}
\usetikzlibrary{shapes,arrows,matrix, positioning}
\pagestyle{empty}
\begin{document}
\tikzset{
decision/.style = {diamond, draw, fill=blue!20,
                   text width=4.5em, text badly centered, node distance=3cm, inner sep=0pt},
block/.style    = {rectangle, draw, fill=blue!20,
                    text width=6cm, text centered, rounded corners, minimum height=4em},
suggestion/.style    = {rectangle, draw, fill=green!20,
                    text width=6cm, text centered, rounded corners, minimum height=4em},
line/.style     = {draw, -latex'},
cloud/.style    = {draw, ellipse,fill=red!20, text width=4.5em, text badly centered,  node distance=3cm,
                   minimum height=2em}
}

\title{A brief guide to the numerical scheme implemented in Basilisk}
\author{Jose Lopez-Herrera \\ Universidad de Sevilla}
\maketitle
\chapter{Introduction}
\section{Equations}
The solved equations are the Navier-Stokes incompressible one given by,
\begin{gather}
\nabla \cdot \mathbf{u} = 0 \\
\partial_t \mathbf{u} + \mathbf{u} \cdot \nabla \mathbf{u} =  \frac{1}{\rho} \left(- \nabla p + \nabla \cdot ( 2\mu \mathcal{D}) \right) + \mathbf{a}
\end{gather}
where $\mathcal{D}$ is the deformation tensor and $\mathbf{a}$ stands for extra acceleration than can be originated by several causes, as for example the gravitational forces, surface tension forces, viscoelastic forces, etc.

Simultaneously it can be computed any tracer, $c_i$ or VOF tracer $T_i$,
\begin{gather}
\partial_t T_i + \mathbf{u} \cdot \nabla T_i = 0 \\
\partial_t c_i +   \nabla \cdot( \mathbf{u} c_i) = 0
\end{gather}
Observe that these tracer can be passively advected with the fluid or, in a more common (and interesting situation), can affect actively the fluid motion as for example by altering the distribution of the fluid properties
\begin{equation}
\rho = \rho (T_i,c_i) \quad \mbox{and} \quad \mu = \mu (T_i,c_i).
\end{equation}
\section{Momentum equation}
The advection part of momentum $(\mathbf{u} \cdot \nabla) \mathbf{u}$ can be more easily computed profiting the incompressibility of the fluid.  In effect, the divergence of the dyadic product $\mathbf{u}\mathbf{u}$ (also written as $\mathbf{u} \bigotimes \mathbf{u}$) writes
$$
\nabla \cdot(\mathbf{u}\mathbf{u}) = \mathbf{u} (\nabla \cdot \mathbf{u}) + (\mathbf{u} \cdot \nabla) \mathbf{u} = (\mathbf{u} \cdot \nabla) \mathbf{u}.
$$

Observe that the Gauss's theorem allows its computation by evaluating the fluxes through the cell's faces,
$$
\int_\Omega \nabla \cdot(\mathbf{u}\mathbf{u}) \, d\Omega = \int_\Sigma \mathbf{u}(\mathbf{u} \cdot \mathbf{n}) \, d\Sigma
$$
This conservative approach of the Gauss's theorem, i.e. to calculate temporal evolution at the cell center by computing fluxes across cell faces is thoroughly used in the numerical scheme.

\newpage
\newgeometry{margin = 10mm}
\begin{landscape}
  \begin{center}
    {\Large \textbf{Flowchart of the generic time step}}
    \resizebox{0.8\linewidth}{!}{
    \begin{tikzpicture}[node distance = 2cm, auto, font=\sffamily\footnotesize]
      % Place nodes
      \node [block] (init) {\textbf{Starting point} \\
        Velocities: (a)centered: $\mathbf{u}^n$ and (b) normal face: $u^{f,n}$  \\
        Centered pressure: $p^n$ \\
        Centered pressure gradient + extra-accelerations (no viscosity):
        $$\mathbf{g}^n = -\nabla p^n + \mathbf{a}^n$$
        Centered tracers: $c_i^{n-1/2}$ \\ VOF tracers: $T_i^{n-1/2}$ };
      \node [block, below of=init, node distance=35mm] (dtmax) {\textbf{event dtmax} \\
        Set the maximum imposed step size (if any)};
      \node [block, below of=dtmax, node distance=25mm] (stability) {\textbf{event Stability} \\
        Reduces time step if necessary according to CFL or other criteria. };
      \node [block, below of=stability, node distance=25mm] (VOF) {\textbf{event VOF} \\
        VoF tracers are advected to time $n+1/2$:
        $$
        T_i^{n+1/2} = T_i^{n-1/2} + \Delta t \nabla \cdot (T^{n}_{i} \mathbf{u}^n)
        $$
      };
      \node [block, below of=VOF, node distance=35mm] (tracers) {\textbf{event tracer\_advection} \\ %\\
        %    Tracers are advected also to time $n+1/2$.
        Using the Bell-Colella-Glaz scheme are advanced the center values of the tracers to the instant $n+1/2$. This method is essentially an extrapolation from center values to face value and from time $n-1/2$ to $n$,
        $c^{f,n}_{i} = \mathbf{f}(c^{n-1/2}_{i}, u^{f,n})$, followed by an explicit update,
        \begin{equation*}
          \begin{split}
            c_i^{n+1/2} &= c_i^{n-1/2} + \Delta t \nabla \cdot (c^{n}_{i} \mathbf{u}^{n}) \\ &= c_i^{n-1/2} +\frac{\Delta t}{h^2} \int_{\partial \mathcal{C}} c^{f,n}_{i} u^{f,n} \, d\Sigma
          \end{split}
        \end{equation*}
      };
      \node [suggestion, below of=tracers, node distance=35mm] (diffusion) {Not present at the moment but I wonder it not would be convenient to add a: \\
        \textbf{event tracer\_diffusion} here}; %\\
      \node [block, right of=init , node distance=9cm] (properties) {\textbf{event properties} \\
        The relevant properties like viscosity and density are advanced to the step $n+1/2$. Typically depend on the tracers,
        $$
        \rho^{n+1/2}, \mu^{n+1/2} = \mathbf{f}(T_i^{n+1/2},c_i^{n+1/2})
        $$
      };
      \node [block, below of=properties, node distance=8cm] (advection) {\textbf{event advection\_term} \\ %\\
        %    Tracers are advected also to time $n+1/2$.
        With a variant of the BCG scheme are calculated normal face velocities at instant $n+1/2$,
        $$
        u_p^{f,n+1/2} = \mathbf{f}(\mathbf{u}^n,\mathbf{g}^n).
        $$
        Note that the extrapolation takes into account either the pressure gradient and extra-accelerations by means of $\mathbf{g}^n$. The label  $p$ denotes that it is a \textsc{prediction}. i.e, its solenoidal character is not guaranteed ($\nabla \cdot \mathbf{u}_p \neq 0$).\\[7pt]

        \textsc{Projection step:} Consist in correct the predicted face velocity $u_p^{f,n+1/2}$ to make it solenoidal. To this end we derive a correction for the face velocities,
        \begin{equation*}
          \begin{split}
            u^{f,n+1/2} &=u_p^{f,n+1/2}+\Delta u^f \\&= u_p^{f,n+1/2}-\nabla p_f (\Delta t)/\rho
          \end{split}
        \end{equation*}
        Attention, $p_f$ is a correction NOT the real pressure field. \\[7pt] % being the solution of,
        %$\nabla \cdot \mathbf{u}_p = \Delta t\nabla \cdot( \nabla p_f/\rho)$
        Advection term is computed using again BCG\footnote{\tiny Note that $\mathbf{u}^{f,n+1/2}$ is the velocity vector at faces while $u^{f,n+1/2}$ is the component \emph{normal} to the face, relevant for calculation of fluxes.},
        \begin{equation*}
          \begin{array}{c}
            \mathbf{u}^{f,n+1/2} = \mathbf{f}(\mathbf{u}^{n},\mathbf{g}^{n}, u^{f,n+1/2}) \\
            \frac{\mathbf{u}_a^{n+1} - \mathbf{u}^{n}}{\Delta t} = \frac{1}{h^2}\int_{\partial \mathcal{C}} u^{f,n+1/2}\mathbf{u}^{f,n+1/2} \, d \Sigma.
          \end{array}
        \end{equation*}
      };
      \node [block, right of=init, node distance=18cm] (viscous) {\textbf{event viscosity} \\
        It solves implicitly the equation
        \begin{equation*}
          \begin{split}
            \frac{\mathbf{u}^{n+1}_v - \mathbf{u}^{n+1}_a}{\Delta t}  = \nabla \cdot [2 \mu \mathcal{D}_v^{n+1}] -\frac{\nabla p^n}{\rho}+\mathbf{a}^n
          \end{split}
        \end{equation*}
        $\mathbf{u}^{n+1}_a$ is the velocity after considering the advection term.\\
        Due to \emph{correction(-$\Delta t$)},
        the center velocity at the end of this event is
        $$
        \mathbf{u}_*^{n+1} = \mathbf{u}^{n+1}_v - \Delta t (-\frac{\nabla p^n}{\rho}+\mathbf{a}^n)
        $$
        Normal face velocity at instant $n+1$, $u_*^{f,n+1}$, are constructed averaging $\mathbf{u}_*^{n+1}$.
      };
      \node [block, below of=viscous, node distance = 5cm] (acceleration) {\textbf{event acceleration} \\ Once the accelerations at faces are calculated at the instant $n+1$, $a^{f,n+1}$, the face velocity is consequently augmented
        $$
        u_{**}^{f,n+1} = u^{f,n+1}_* + \Delta t a^{f,n+1}
        $$
      };

      \node [block, below of=acceleration, node distance=6cm] (projection) {\textbf{event projection} \\ A projection step with $u_{**}^{f,n+1}$ is done. As a result a normal face velocities are divergence free
        $$
        u^{f,n+1} = u_{**}^{f,n+1} -\nabla p^{n+1} (\Delta t)/\rho
        $$
        The correction pressure $p^{n+1}$ is considered also the pressure field at instant $n+1$.

        Face values of the enriched normal pressure gradient $g^{f,n+1}$ are calculated from the pressure field $p^{n+1}$ and acceleration $a^{f,n+1}$. Centered value $\mathbf{g}^{n+1}$ are calculated simply by averaging.

        Finally, centered velocity at $n+1$ is simply,
        $$
        \mathbf{u}^{n+1} = \mathbf{u}_{*}^{n+1} + \Delta t [-(\nabla p^{n+1})/\rho +\mathbf{a}^{n+1}]
        $$
      };
      \node [cloud, below of=projection, node distance=4cm] (end) {End timestep};
      % Draw edges
      \path [line] (init) -- (dtmax);
      \path [line] (dtmax) -- (stability);
      \path [line] (stability) -- (VOF);
      \path [line] (VOF) -- (tracers);
      \path [line] (tracers) -- (diffusion);
      \path [line] (diffusion.east) -| ([xshift=-1.5cm] properties.west)
      -- (properties.west);
      \path [line] (properties) -- (advection);
      %    \path [line] (advection) -- (viscous);
      \path [line] (advection.east) -| ([xshift=-1.5cm] viscous.west)
      -- (viscous.west);
      \path [line] (viscous) -- (acceleration);
      \path [line] (acceleration) -- (projection);
      \path [line] (projection) -- (end);
    \end{tikzpicture}
}
  \end{center}
\end{landscape}
\newpage
\restoregeometry
\section{Additional considerations}

The basilisk uses a fractional step method consisting in:
\begin{equation}
\begin{array}{lcrcl}
\mbox{\textbf{event advection}} &  \quad &  \mathbf{u}^{n+1}_a    & = & \mathbf{u}^{n-1} - \Delta t (\mathbf{u}^{n+1/2} \cdot \nabla \mathbf{u}^{n+1/2})  \\
\mbox{\textbf{event viscosity}, \emph{correction($\Delta t$)}} & \quad & \mathbf{u}^{n+1}_r   & = & \mathbf{u}^{n+1}_a + \Delta t (-\nabla p^n/\rho + \mathbf{a}^{n})  \\
\mbox{\textbf{event viscosity}, \emph{viscous()}} & \quad & \mathbf{u}^{n+1}_v   & = & \mathbf{u}^{n+1}_r + \Delta t (\nabla \cdot [2 \mu \mathcal{D}^{n+1}_v])  \\
\mbox{\textbf{event viscosity}, \emph{correction(-$\Delta t$)}} & \quad & \mathbf{u}^{n+1}_*   & = & \mathbf{u}^{n+1}_v - \Delta t (-\nabla p^n/\rho + \mathbf{a}^{n})  \\
\mbox{\textbf{event projection}} & \quad & \mathbf{u}^{n+1}   & = & \mathbf{u}^{n+1}_* + \Delta t (-\nabla p^{n+1}/\rho + \mathbf{a}^{n+1})  \\
\end{array}
\end{equation}
Summing up all this fractional steps, the general scheme is,
\begin{equation}
\mathbf{u}^{n+1}  =  \mathbf{u}^{n} + \Delta t (-\mathbf{u}^{n+1/2} \cdot \nabla \mathbf{u}^{n+1/2}-\nabla p^{n+1}/\rho + \nabla \cdot [2 \mu \mathcal{D}^{n+1}_v] + \mathbf{a}^{n+1})
\end{equation}
being the deformation tensor $\mathcal{D}$ calculated with the velocity, $\mathbf{u}_v^{n+1}$, resulting of the implicit approximate equation,
\begin{equation}
\mathbf{u}_v^{n+1}  =  \mathbf{u}^{n} + \Delta t (-\mathbf{u}^{n+1/2} \cdot \nabla \mathbf{u}^{n+1/2}-\nabla p^{n}/\rho + \nabla \cdot [2 \mu \mathcal{D}^{n+1}_v] + \mathbf{a}^{n})
\end{equation}

Since the acceleration event is located after the viscosity event, in case that the acceleration depend upon the velocity it would be,
$$
\mathbf{a}^{n+1} = \mathbf{a}^{n+1} (\mathbf{u}_*^{n+1})
$$

{\color{red} Should not be the acceleration being calculated with $\mathbf{u}_v^{n+1}$ instead of $\mathbf{u}_*^{n+1}$ ? }


\end{document}
